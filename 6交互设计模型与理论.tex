\subsubsection*{\S 交互设计模型与理论}
\setcounter{problemname}{0}

\begin{problem}
	‍‌以下关于任务分析描述错误的是:
	\uline{C}    
    %\vspace{-0.8em}
    %\begin{multicols}{4}
        \begin{enumerate}[label=\Alph*.]
            \item 层次化任务分析采用的是“分而治之”的方法
            \item 任务分析对于改善用户体验至关重要
            \item 只要肯花时间,总是可以实现完善的任务分析
            \item 层次化任务分析是人因工效学领域中最广泛使用的方法
        \end{enumerate}
    %\end{multicols}
    %\vspace{-1em}
\end{problem}

\begin{solution}
C.任务分析没有统一的标准答案,因而也很难定义什么样的任务分析是完善的。
\end{solution}


\begin{problem}
	GOMS的全称是什么?
	\uline{A}    
    \vspace{-0.8em}
    \begin{multicols}{2}
        \begin{enumerate}[label=\Alph*.]
            \item Goals, operation, methods and selection rules
            \item Goals, objects, models and selection rules
            \item Goals, operations, methods and state rules
            \item Goals, operations, models and state rules
        \end{enumerate}
    \end{multicols}
    \vspace{-1em}
\end{problem}

\begin{solution}
A.目标、操作、方法和选择规则。
\end{solution}


\begin{problem}
	以下关于击键层次模型描述不正确的是:
	\uline{B}    
    %\vspace{-0.8em}
    %\begin{multicols}{2}
        \begin{enumerate}[label=\Alph*.]
            \item 击键层次模型预测假设交互过程中没有错误发生
            \item 使用击键层次模型预测的难点在于对操作路径的分析
            \item 击键层次模型用于预测指点任务的完成时间
            \item 击键层次模型预测的是无干扰情况下完成任务的时间
        \end{enumerate}
    %\end{multicols}
    %\vspace{-1em}
\end{problem}

\begin{solution}
B.使用击键层次模型的难点在于M操作符的使用,即在哪些操作之前需要引入一个思维过程。
\end{solution}


\begin{problem}
	以下关于Fitts定律描述不正确的是:
	\uline{C}    
    %\vspace{-0.8em}
    %\begin{multicols}{2}
        \begin{enumerate}[label=\Alph*.]
            \item Fitts定律对于图形用户界面应用开发具有重要指导意义
            \item Fitts定律也可用于指导现实生活中的产品设计
            \item Fitts定律可以预测任意交互操作的完成时间
            \item Fitts定律是一种预测模型
        \end{enumerate}
    %\end{multicols}
    %\vspace{-1em}
\end{problem}

\begin{solution}
C. Fitts定律只能够预测指点型任务的完成时间,即应用某种操作设备(鼠标、触摸板、手指等)访问屏幕某个区域或对象的时间。
\end{solution}


\begin{problem}
	以下关于预测模型描述正确的是:
	\uline{A}    
    %\vspace{-0.8em}
    %\begin{multicols}{2}
        \begin{enumerate}[label=\Alph*.]
            \item 预测模型可用于比较不同的应用软件和设备
            \item 预测模型只能预测任务的完成时间
            \item 预测模型预测的任务完成时间和实际用户的任务执行时间一致
            \item 预测模型能够对所有任务的完成情况进行预测
        \end{enumerate}
    %\end{multicols}
    %\vspace{-1em}
\end{problem}

\begin{solution}
B.预测模型不仅能够预测任务的完成时间,还能对任务的完成策略(如次序)进行预测。  C.预测模型预测的是专家用户在没有任何错误的情况下完成任务的时间,通常与真实用户的实际任务执行时间存在出入。  D.预测模型只能预测“认知-动作型”任务的完成情况。
\end{solution}