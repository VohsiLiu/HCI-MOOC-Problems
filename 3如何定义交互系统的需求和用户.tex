\subsubsection*{\S 如何定义交互系统的需求和用户}
\setcounter{problemname}{0}

\begin{problem}
	‌以下哪一条是针对专家用户的设计原则:
	\uline{A}    
    \vspace{-0.8em}
    \begin{multicols}{2}
        \begin{enumerate}[label=\Alph*.]
            \item 确保快速的响应时间
            \item 使用含义丰富的信息
            \item 减轻记忆负担
            \item 提供说明、对话框和在线帮助
        \end{enumerate}
    \end{multicols}
    \vspace{-1em}
\end{problem}

\begin{solution}
A.专家用户通常可以为了获得比较快的交互效率而牺牲一些记忆负担,比如快捷键、热键的使用。  B.这一条本身没有错,但它不是针对专家用户而言,对普通用户更需要使用含义丰富的信息。  C.这一条对于所有用户都适用,没有人喜欢复杂且难以记住的交互。  D.这一条明显针对的是不熟悉的新手用户。
\end{solution}


\begin{problem}
	以下关于短时记忆描述正确的是:
	\uline{A}    
    \vspace{-0.8em}
    \begin{multicols}{2}
        \begin{enumerate}[label=\Alph*.]
            \item 短时记忆的容量是有限的
            \item 短时记忆的容量很大,但容量有限
            \item 短时记忆的容量为零
            \item 短时记忆的容量是无限的
        \end{enumerate}
    \end{multicols}
    \vspace{-1em}
\end{problem}

\begin{solution}
A.短期记忆有限,但是人类信息加工系统的核心;长时记忆的容量是无限的,且存入的信息不会消失。
\end{solution}


\begin{problem}
	在进行界面设计时要注意对组件进行对齐,这是由于格式塔心理学中哪一条原则的影响:
	\uline{A}    
    \vspace{-0.8em}
    \begin{multicols}{4}
        \begin{enumerate}[label=\Alph*.]
            \item 连续性
            \item 完整性
            \item 相似性
            \item 相近性
        \end{enumerate}
    \end{multicols}
    \vspace{-1em}
\end{problem}

\begin{solution}
A.连续性表示人类视觉更习惯沿着一个平滑的方向去观察事物,没有对齐的组件会增大用户的认知负担。
\end{solution}


\begin{problem}
	假设需要判断某应用程序的配色方案是否恰当。对于该测试任务,您将使用:
	\uline{A}    
    \vspace{-0.8em}
    \begin{multicols}{2}
        \begin{enumerate}[label=\Alph*.]
            \item 高保真模型
            \item 低保真模型
            \item 低保真和高保真模型均可
            \item 以上都错
        \end{enumerate}
    \end{multicols}
    \vspace{-1em}
\end{problem}

\begin{solution}
A.题目问的是配色方案的选择,只有在高保真原型上才能对颜色、字体、视觉元素等比较具体的交互细节进行评估。
\end{solution}


\begin{problem}
	以下关于“人物角色”描述正确的是:
	\uline{A}    
    %\vspace{-0.8em}
    %\begin{multicols}{2}
        \begin{enumerate}[label=\Alph*.]
            \item 人物角色能够帮助克服当前数字产品开发中的很多问题
            \item 人物角色的概念以及使用都很简单
            \item 人物角色是设计人员编造的,并不是真实的人
            \item 人物角色不是特定于上下文的,所以它可以很容易地被重复使用
        \end{enumerate}
    %\end{multicols}
    %\vspace{-1em}
\end{problem}

\begin{solution}
A.人物角色有助于解决设计中的弹性用户、自参考设计、边缘情况设计等一系列问题。  B.人物角色的概念比较简单,但人物角色的构造及使用并不简单。  C.人物角色不是真实的人,但是在真实的用户调研的基础上得到的,并非由设计人员编造。  D.人物角色是基于对特定产品的用户调研基础上得到的,不同产品的用户群体和特点都不相同,因此人物角色不能被简单地重复使用。
\end{solution}