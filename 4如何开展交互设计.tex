\subsubsection*{\S 如何开展交互设计}
\setcounter{problemname}{0}

\begin{problem}
	以下关于Allan Cooper提出的交互设计框架描述正确的是:
	\uline{C}    
    %\vspace{-0.8em}
    %\begin{multicols}{2}
        \begin{enumerate}[label=\Alph*.]
            \item 验证性的场景剧本需要具备产品的很多细节信息
            \item 建议使用高保真草图序列的故事板来描述关键线路情景剧本
            \item 关键线路情景剧本必须在细节上严谨地描述每个主要交互的精确行为
            \item 交互设计框架可用于确定界面使用的颜色和风格
        \end{enumerate}
    %\end{multicols}
    %\vspace{-1em}
\end{problem}

\begin{solution}
A.验证性场景剧本的重点是全面覆盖用户可能的交互路径,但不用具备很多界面细节。  B.关键线路情景剧本描述的是人物角色最频繁使用界面的主要路径,此时仍在设计的初级阶段,建议使用低保真的故事板。  D.交互设计框架解决的是产品高层次上的屏幕布局和工作流等,而不是界面颜色等风格等非常具体的视觉问题。
\end{solution}


\begin{problem}
	原型阶段跟在哪一个开发阶段的后面?
	\uline{C}    
    \vspace{-0.8em}
    \begin{multicols}{4}
        \begin{enumerate}[label=\Alph*.]
            \item 评估
            \item 构建应用程序
            \item 理解用户需要
            \item 以上都不对
        \end{enumerate}
    \end{multicols}
    \vspace{-1em}
\end{problem}

\begin{solution}
C.通常建议在对用户获取需求之后,通过原型快速检验需求获取的正确性。
\end{solution}


\begin{problem}
	假设需要判断某应用程序的配色方案是否恰当。对于该测试任务,您将使用:
	\uline{A}    
    \vspace{-0.8em}
    \begin{multicols}{2}
        \begin{enumerate}[label=\Alph*.]
            \item 高保真模型
            \item 低保真模型
            \item 低保真模型和高保真模型均可
            \item 低保真模型和高保真模型均不合适
        \end{enumerate}
    \end{multicols}
    \vspace{-1em}
\end{problem}

\begin{solution}
A.题目问的是配色方案的选择,只有在高保真原型上才能对颜色、字体、视觉元素等比较具体的交互细节进行评估。
\end{solution}


\begin{problem}
	‍对于主流用户很少使用,但自身需要更新的功能,可使用何种策略进行简化:
	\uline{B}    
    \vspace{-0.8em}
    \begin{multicols}{4}
        \begin{enumerate}[label=\Alph*.]
            \item 转移
            \item 隐藏
            \item 删除
            \item 组织
        \end{enumerate}
    \end{multicols}
    \vspace{-1em}
\end{problem}

\begin{solution}
A.转移主要考虑的是在设备间以及在设备和用户之间的功能分配问题。  C.删除适合针对那些用户不需要的功能。  D.对于界面上确认保留的功能建议都要进行合理地组织。
\end{solution}


\begin{problem}
	‍关于交互设计模式,以下说法错误的是:
	\uline{D}    
    \vspace{-0.8em}
    \begin{multicols}{2}
        \begin{enumerate}[label=\Alph*.]
            \item 模式捕捉了良好设计中不变的特性
            \item 模式在交互设计中的应用还处于起步阶段
            \item 设计模式能帮助提供有价值、有用的设计思路
            \item 交互设计模式可以拿来即用,不需要修改
        \end{enumerate}
    \end{multicols}
    \vspace{-1em}
\end{problem}

\begin{solution}
D.模式捕捉的只是良好设计中不变的特性,其具体实现取决于特定场景和设计者的创造性。模式并不是拿来即用的商品。
\end{solution}