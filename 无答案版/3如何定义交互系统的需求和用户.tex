\subsubsection*{\S 如何定义交互系统的需求和用户}
\setcounter{problemname}{0}

\begin{problem}
	‌以下哪一条是针对专家用户的设计原则:
    \vspace{-0.8em}
    \begin{multicols}{2}
        \begin{enumerate}[label=\Alph*.]
            \item 确保快速的响应时间
            \item 使用含义丰富的信息
            \item 减轻记忆负担
            \item 提供说明、对话框和在线帮助
        \end{enumerate}
    \end{multicols}
    \vspace{-1em}
\end{problem}



\begin{problem}
	以下关于短时记忆描述正确的是:  
    \vspace{-0.8em}
    \begin{multicols}{2}
        \begin{enumerate}[label=\Alph*.]
            \item 短时记忆的容量是有限的
            \item 短时记忆的容量很大,但容量有限
            \item 短时记忆的容量为零
            \item 短时记忆的容量是无限的
        \end{enumerate}
    \end{multicols}
    \vspace{-1em}
\end{problem}



\begin{problem}
	在进行界面设计时要注意对组件进行对齐,这是由于格式塔心理学中哪一条原则的影响:
    \vspace{-0.8em}
    \begin{multicols}{4}
        \begin{enumerate}[label=\Alph*.]
            \item 连续性
            \item 完整性
            \item 相似性
            \item 相近性
        \end{enumerate}
    \end{multicols}
    \vspace{-1em}
\end{problem}



\begin{problem}
	假设需要判断某应用程序的配色方案是否恰当。对于该测试任务,您将使用:  
    \vspace{-0.8em}
    \begin{multicols}{2}
        \begin{enumerate}[label=\Alph*.]
            \item 高保真模型
            \item 低保真模型
            \item 低保真和高保真模型均可
            \item 以上都错
        \end{enumerate}
    \end{multicols}
    \vspace{-1em}
\end{problem}



\begin{problem}
	以下关于“人物角色”描述正确的是:
    %\vspace{-0.8em}
    %\begin{multicols}{2}
        \begin{enumerate}[label=\Alph*.]
            \item 人物角色能够帮助克服当前数字产品开发中的很多问题
            \item 人物角色的概念以及使用都很简单
            \item 人物角色是设计人员编造的,并不是真实的人
            \item 人物角色不是特定于上下文的,所以它可以很容易地被重复使用
        \end{enumerate}
    %\end{multicols}
    %\vspace{-1em}
\end{problem}

