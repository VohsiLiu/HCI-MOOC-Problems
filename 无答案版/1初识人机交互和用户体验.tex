\subsubsection*{\S 初识人机交互和用户体验}
\setcounter{problemname}{0}

\begin{problem}
	‍在人机交互领域,计算机可能指的是:
    \vspace{-0.8em}
    \begin{multicols}{4}
        \begin{enumerate}[label=\Alph*.]
            \item 台式机
            \item 大型计算机系统
            \item 网站
            \item 以上都对
        \end{enumerate}
    \end{multicols}
    \vspace{-1em}
\end{problem}




\begin{problem}
	‍以下描述正确的是: 
    \vspace{-0.8em}
    \begin{multicols}{2}
        \begin{enumerate}[label=\Alph*.]
            \item 人机交互只关注软件的可用性
            \item 人机交互就是用户界面设计
            \item “以用户为中心”是交互设计的主要方法
            \item 人机交互只需关注软件设计,不需要关注用户
        \end{enumerate}
    \end{multicols}
    \vspace{-1em}
\end{problem}


\begin{problem}
	以下哪个领域不会对人机交互学科产生影响?
    \vspace{-0.8em}
    \begin{multicols}{2}
        \begin{enumerate}[label=\Alph*.]
            \item 计算机科学
            \item 人因功效学
            \item 认知心理学
            \item 上述学科均对人机交互学科有影响
        \end{enumerate}
    \end{multicols}
    \vspace{-1em}
\end{problem}



\begin{problem}
	人机交互是交叉学科,作为交叉学科团队的主要缺点是: 
    \vspace{-0.8em}
    \begin{multicols}{2}
        \begin{enumerate}[label=\Alph*.]
            \item 会产生过多想法
            \item 看待和谈论问题的角度不同
            \item 相互沟通不容易
            \item 以上都不是
        \end{enumerate}
    \end{multicols}
    \vspace{-1em}
\end{problem}



\begin{problem}
	在 EEC 模型中,用户为达目标而制定的动作与系统允许的动作之间的差别被称作:  
    \vspace{-0.8em}
    \begin{multicols}{4}
        \begin{enumerate}[label=\Alph*.]
            \item 执行阶段
            \item 评估阶段
            \item 执行隔阂
            \item 评估隔阂
        \end{enumerate}
    \end{multicols}
    \vspace{-1em}
\end{problem}
