\subsubsection*{\S 如何对交互性能进行评价}
\setcounter{problemname}{0}

\begin{problem}
	‍以下关于交互评估描述错误的是:
    \vspace{-0.8em}
    \begin{multicols}{2}
        \begin{enumerate}[label=\Alph*.]
            \item 评估过程需要严谨的设计
            \item 评估是设计过程中一个独立的阶段
            \item 评估不一定要遵循DECIDE框架
            \item 评估是系统化的数据搜集过程
        \end{enumerate}
    \end{multicols}
    \vspace{-1em}
\end{problem}



\begin{problem}
	‍为探索孩子们在一起是如何交谈的,并调查一种新型产品是否能帮助他们更积极地参与其中,可使用如下哪种技术:
    \vspace{-0.8em}
    \begin{multicols}{4}
        \begin{enumerate}[label=\Alph*.]
            \item 实地研究
            \item 预测性评估
            \item 可用性测试
            \item DECIDE框架
        \end{enumerate}
    \end{multicols}
    \vspace{-1em}
\end{problem}



\begin{problem}
	‍关于启发式评估,以下论述正确的是:
    %\vspace{-0.8em}
    %\begin{multicols}{4}
        \begin{enumerate}[label=\Alph*.]
            \item 启发式评估是一种基于专家的评估方法
            \item 启发式评估的结果只有界面中潜在的可用性问题列表
            \item 当界面元素存在多个可用性问题时,只需列举其中一个问题即可
            \item 专家应用启发式评估时,会从自身使用经验出发对界面进行判断
        \end{enumerate}
    %\end{multicols}
    %\vspace{-1em}
\end{problem}



\begin{problem}
	‍以下哪一条不属于用户测试前的准备步骤: 
    \vspace{-0.8em}
    \begin{multicols}{4}
        \begin{enumerate}[label=\Alph*.]
            \item 制定测试方案
            \item 选择参与用户
            \item 开展小规模测试
            \item 观察参与者
        \end{enumerate}
    \end{multicols}
    \vspace{-1em}
\end{problem}



\begin{problem}
	‍‍以下论述正确的是:
    %\vspace{-0.8em}
    %\begin{multicols}{4}
        \begin{enumerate}[label=\Alph*.]
            \item 原型既可以帮助发现设计问题,也可以用来帮助用户明确需求
            \item 评估不应该过早进行,因为此时系统还不够完善
            \item 高保真原型更接近系统,因而在评估中要尽可能使用高保真原型进行评估
            \item 纸质原型适用于产品开发过程中的任意阶段
        \end{enumerate}
    %\end{multicols}
    %\vspace{-1em}
\end{problem}
