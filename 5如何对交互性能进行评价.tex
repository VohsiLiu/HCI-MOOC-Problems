\subsubsection*{\S 如何对交互性能进行评价}
\setcounter{problemname}{0}

\begin{problem}
	‍以下关于交互评估描述错误的是:
	\uline{B}    
    \vspace{-0.8em}
    \begin{multicols}{2}
        \begin{enumerate}[label=\Alph*.]
            \item 评估过程需要严谨的设计
            \item 评估是设计过程中一个独立的阶段
            \item 评估不一定要遵循DECIDE框架
            \item 评估是系统化的数据搜集过程
        \end{enumerate}
    \end{multicols}
    \vspace{-1em}
\end{problem}

\begin{solution}
B.优秀的交互设计师应掌握如何在不同的开发阶段对系统的不同形式进行评估,因此不应将评估作为某个开发节点上的独立阶段。
\end{solution}


\begin{problem}
	‍为探索孩子们在一起是如何交谈的,并调查一种新型产品是否能帮助他们更积极地参与其中,可使用如下哪种技术:
	\uline{A}    
    \vspace{-0.8em}
    \begin{multicols}{4}
        \begin{enumerate}[label=\Alph*.]
            \item 实地研究
            \item 预测性评估
            \item 可用性测试
            \item DECIDE框架
        \end{enumerate}
    \end{multicols}
    \vspace{-1em}
\end{problem}

\begin{solution}
A.实地研究特别适合在真实环境中去理解用户的实际工作情形以及技术对他们的影响,特别适合于面向儿童和残疾人等特殊用户群体的理解和对产品开展评估工作。
\end{solution}


\begin{problem}
	‍关于启发式评估,以下论述正确的是:
	\uline{A}    
    %\vspace{-0.8em}
    %\begin{multicols}{4}
        \begin{enumerate}[label=\Alph*.]
            \item 启发式评估是一种基于专家的评估方法
            \item 启发式评估的结果只有界面中潜在的可用性问题列表
            \item 当界面元素存在多个可用性问题时,只需列举其中一个问题即可
            \item 专家应用启发式评估时,会从自身使用经验出发对界面进行判断
        \end{enumerate}
    %\end{multicols}
    %\vspace{-1em}
\end{problem}

\begin{solution}
B.启发式评估专家不仅能够罗列出产品中潜在的可用性问题,同时还能够针对问题给出建设性的修改意见和建议。  C.在开展启发式评估时应该尽可能罗列出所有存在的问题以及其违反的启发式原则。  D.在启发式评估中,专家应该使用基于角色扮演的方式去模拟目标用户使用产品的感受。
\end{solution}


\begin{problem}
	‍以下哪一条不属于用户测试前的准备步骤:
	\uline{D}    
    \vspace{-0.8em}
    \begin{multicols}{4}
        \begin{enumerate}[label=\Alph*.]
            \item 制定测试方案
            \item 选择参与用户
            \item 开展小规模测试
            \item 观察参与者
        \end{enumerate}
    \end{multicols}
    \vspace{-1em}
\end{problem}

\begin{solution}
D.用户测试的目的是对待测产品进行分析和评价,因而对实验人员的观察并不属于测试的相关工作。
\end{solution}


\begin{problem}
	‍‍以下论述正确的是:
	\uline{A}    
    %\vspace{-0.8em}
    %\begin{multicols}{4}
        \begin{enumerate}[label=\Alph*.]
            \item 原型既可以帮助发现设计问题,也可以用来帮助用户明确需求
            \item 评估不应该过早进行,因为此时系统还不够完善
            \item 高保真原型更接近系统,因而在评估中要尽可能使用高保真原型进行评估
            \item 纸质原型适用于产品开发过程中的任意阶段
        \end{enumerate}
    %\end{multicols}
    %\vspace{-1em}
\end{problem}

\begin{solution}
B.即便在还没有完成系统开发时,应用快速评估等方法去获取目标用户关于产品的意见和建议都是非常有益的。  C.高保真原型一方面由于制作较为困难,另一方面因其过于接近目标系统,从而会让用户误以为已经开发完成从而不再提出一些原则性的修改意见,同时开发人员也会误认为已找到了一个用户满意的设计,因而不再考虑其它方案。  D.通常在实际可运行的产品开发出来之后,就不建议继续使用纸质原型进行评价了。
\end{solution}