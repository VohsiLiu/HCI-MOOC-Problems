\subsubsection*{\S 初识人机交互和用户体验}
\setcounter{problemname}{0}

\begin{problem}
	‍在人机交互领域,计算机可能指的是:
	\uline{D}    
    \vspace{-0.8em}
    \begin{multicols}{4}
        \begin{enumerate}[label=\Alph*.]
            \item 台式机
            \item 大型计算机系统
            \item 网站
            \item 以上都对
        \end{enumerate}
    \end{multicols}
    \vspace{-1em}
\end{problem}

\begin{solution}
D.HCI中的C指的是interactive computing devices,泛指一切可计算的设备和系统。
\end{solution}


\begin{problem}
	‍以下描述正确的是:
	\uline{C}    
    \vspace{-0.8em}
    \begin{multicols}{2}
        \begin{enumerate}[label=\Alph*.]
            \item 人机交互只关注软件的可用性
            \item 人机交互就是用户界面设计
            \item “以用户为中心”是交互设计的主要方法
            \item 人机交互只需关注软件设计,不需要关注用户
        \end{enumerate}
    \end{multicols}
    \vspace{-1em}
\end{problem}

\begin{solution}
A.人机交互不仅关注系统是否能正确执行其功能(即功能正确性),还希望系统能够具有较好的易学性、易用性等特点(即可用性),同时还应该带给用户较正面的主观感受(即广义的用户体验等)。  B.HCI中的I从interface转变到如今的interaction,其内涵已经不仅仅是用户与之交互的界面,同时可能影响到系统的输入输出的选择、数据的组织方式、甚至系统架构等。  D.引用Don Norman的一句话,``Remember that HCI has an `H' in it: H for Human. H for Humanity."
\end{solution}


\begin{problem}
	以下哪个领域不会对人机交互学科产生影响?
	\uline{D}    
    \vspace{-0.8em}
    \begin{multicols}{2}
        \begin{enumerate}[label=\Alph*.]
            \item 计算机科学
            \item 人因功效学
            \item 认知心理学
            \item 上述学科均对人机交互学科有影响
        \end{enumerate}
    \end{multicols}
    \vspace{-1em}
\end{problem}

\begin{solution}
D. HCI是典型的交叉学科,计算机科学能够提供系统实现所需软硬件技术,人因工效学能够使系统设计符合人的“硬件”构成,认知心理学帮助理解人的认知特点,从而使交互更加人性化。
\end{solution}


\begin{problem}
	人机交互是交叉学科,作为交叉学科团队的主要缺点是:
	\uline{C}    
    \vspace{-0.8em}
    \begin{multicols}{2}
        \begin{enumerate}[label=\Alph*.]
            \item 会产生过多想法
            \item 看待和谈论问题的角度不同
            \item 相互沟通不容易
            \item 以上都不是
        \end{enumerate}
    \end{multicols}
    \vspace{-1em}
\end{problem}

\begin{solution}
C.其他答案也在交叉团队沟通中可能存在的问题,但是最主要的问题还是由于人们往往只擅长以上学科中的某一个或某几个方面,所以大多数情况下并不能理解其他成员使用的所有技术术语,从而出现沟通困难。
\end{solution}


\begin{problem}
	在 EEC 模型中,用户为达目标而制定的动作与系统允许的动作之间的差别被称作:
	\uline{C}    
    \vspace{-0.8em}
    \begin{multicols}{4}
        \begin{enumerate}[label=\Alph*.]
            \item 执行阶段
            \item 评估阶段
            \item 执行隔阂
            \item 评估隔阂
        \end{enumerate}
    \end{multicols}
    \vspace{-1em}
\end{problem}

\begin{solution}
C. EEC模型的意义体现在可以用于解释为什么有些界面会给用户使用带来问题。Norman使用“执行隔阂(gulfs of execution)”和“评估隔阂(gulfs of evaluatoin)”两个术语来描述相关问题,其中系统状态的实际表现与用户预期之间的差别被称作评估隔阂。
\end{solution}